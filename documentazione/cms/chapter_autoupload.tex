\chapter[Auto Upload]{Upload Automatico dei file}

È disponibile uno script per il caricamento automatico delle circolari all'interno del sito. È necessario disporre di una versione aggiornata di OpenOffice (>3.1) per poter usufruire della conversione automatica in pdf dei file di testo.
Il funzionamento delo script è molto semplice: un utente definisce una directory all'interno della quale salvare le circolari (in .odt, .doc, .rtf, .pdf etc) e una directory all'interno della quale salvare i file convertiti, nel momento in cui un file viene salvato all'interno della directory monitorata questo viene convertito in pdf e caricato sul sito nell'opportuna posizione.
\section{Il file di configurazione}

Analiziamo ora il file di configurazione dello script:
\begin{verbatim}
 ---
javapath: java
log_file: C:\Programmi\autopdfconverter\log_upload.log
log_proc: c:\tmp\pdf\Circolari\processed.log
monitor_dir: c:\tmp\Circolari
monitor_sub_dirs: 1
path: c:\tmp\Circolari
pdfconverter: c:/Programmi/autopdfconverter/lib/jodconverter.jar
processed_files: c:/tmp/pdf
timeout_value: 2
users_folders: Pubbliche|Generali
webdav_password: sambackett
webdav_realm: eZ Components WebDAV
webdav_upload_base: Circolari
webdav_url: http://ezdav.hell.pit/ita/Contenuti/
webdav_user: admin
remote_upload: 1
monitor_password: daosdjahsdk
monitor_username: utente
monitor_access_url: c:\tmp
upload_share: \\amrael\mosa\tmp\upload
upload_share_password: rwrwerwerwerwe
upload_share_username: utente
files_db: file_status.csv
localnet_upload: 0
\end{verbatim}
\begin{description}
 \item [javapath] Il path standard di Java nella quasi totalità dei casi non è necessario cambiare questo valore
\item[log\_file] Il nome del file con path assoluto in cui salvare il log dei processi di conversione
\item[log\_proc] Attualmente non utilizzato
\item[monitor\_dir] Il path assoluto della directory all'interno della quale verranno copiate le circolari, può anche essere un UNC
\item[path] Nella versione attuale del codice path e monitor\_dir devono avere lo stesso valore
\item[pdfconverter] Il path assoluto del bridge con OpenOffice
\item[processed\_files] Dove salvare i file pdf prodotti, la struttura delle directory deve essere la stessa presente in monitor\_dir
\item[timeout\_value] Il numero di secondi tra due lavori di gestione consecutivi sui file
\item[users\_folders]Il nome delle directory all'interno della directory monitorata, la stessa gerarchia deve essere presente anche all'interno del sito
\item[webdav\_password] La password dell'utente utilizzato per il caricamenteo delle circolari
\item[webdav\_realm] La stringa identificativa del server webdav remoto NON modificare
\item[webdav\_upload\_base]La directory principale all'interno della quale caricare le circolari
\item[webdav\_url] L'indirizzo del server webdav
\item[webdav\_user] L'utente con i permessi di inserimento dati nella zona circolari
\item[remote\_upload] 1: carica in rete i file convertiti, 0: Ignora questa procedura.
\item[monitor\_password] Password per accedere alla directory monitorata (nel caso in cui questa non si trovi sul file system locale)
\item[monitor\_username] Username per accedere alla directory monitorata (nel caso in cui questa non si trovi nel file system locale)
\item[monitor\_access\_url] Un path qualsiasi all'interno del file system in cui si trova la directory monitorata
\item[upload\_share] L'unc di uno share smb remoto sui cui caricare una copia dei file pdf prodotti
\item[upload\_share\_password]La password dello share smb sulla rete locale
\item[upload\_share\_username] Il nome utente per accedere allo share smb sulla rete locale
\item[files\_db] Nome (senza path) del file in cui salvera la lista dei file caricati
\item[localnet\_upload] 0: non caricare i file su uno share locale,1: carica i file su uno share locale
\end{description}

Procedura per l'installazione dello script:

\begin{enumerate}
\item Installate l'ultima versione di OpenOffice
\item Copiate la directory autopdfconverter in \%programfiles\%  (solitamente c:\backslash Programmi)
\item Copiate la directory utils in  \%programfiles\%  (solitamente c:\backslash Programmi)
\item Eseguite lo script openoffice_service.bat, viene creato un servizio per l'esecuzione di OpenOffice in background
\item Copiate il file processed\_files nella directory impostata alla voce processed\_files all'interno del file di configurazione
\item Eseguite lo script pdf_converter.bat, viene creato un servizio per il monitoraggio in background di una directory da voi scelta
\item Andate su Pannello di controllo - Strumenti di Amministrazione - Servizi, cliccate con il tasto destro sul servizio pdfConverter e visualizzatene le proprietà. Accedete alla tab Connessione della finestra che si è appena aperta  e configurate un utente del sistema con sufficienti permessi in lettura e scrittura nelle directory da voi impostate per le circolari.
\item Riavviate il servizio

\end{enumerate}




