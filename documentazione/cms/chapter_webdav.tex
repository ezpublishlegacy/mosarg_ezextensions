\chapter{Webdav}

\section{Client supportati}



Poiché il client webdav integrato in windows soffre di molti bug è necessario scaricare un
client gratuito alternativo che potete trovare all'indirizzo:

\subsection{Windows::Bitkinex}

http://www.bitkinex.com/download.php
il programma si installa senza particolari complicazioni. 

\begin{figure}[H]
 \centering
 \includegraphics[width=0.5\textwidth]{./immagini/webdav/webdav1.png}
 % webdav1.png: 588x444 pixel, 96dpi, 15.56x11.75 cm, bb=0 0 441 333
\caption{Installazione Bitkinex}\label{fig:BitKinex}
\end{figure}






Alla fine dell'installazione il programma viene eseguito automaticamente e ci compare la
prima finestra di dialogo, se per connettervi ad internet utilizzate un proxy dovrete
specificarlo qui:

  \begin{figure}[H]
 \centering
 \includegraphics[width=\textwidth]{./immagini/webdav/webdav2.png}
 % webdav2.png: 893x558 pixel, 96dpi, 23.62x14.76 cm, bb=
 \caption{Configurazione di BitKinex passo 1}
 \label{fig:bitkinex_1}
\end{figure}



Ci viene poi chiesto se vogliamo configurare una sorgente di dati, assecondiamo il
programma:



\begin{figure}[H]
 \centering
 \includegraphics[width=\textwidth]{./immagini/webdav/webdav3.png}
 % webdav3.png: 896x632 pixel, 96dpi, 23.70x16.72 cm, bb=
 \caption{Configurazione di BitKinex passo 2}
 \label{fig:bitkinex2}
\end{figure}


Ora dovremo inserire i dati della nostra connessione, siccome ci vogliamo collegare via
WebDav utilizzeremo il protocollo http:

\begin{figure}[H]
 \centering
 \includegraphics[width=\textwidth]{./immagini/webdav/webdav4.png}
 % webdav4.png: 591x380 pixel, 96dpi, 15.63x10.05 cm, bb=
 \caption{Configurazione di BitKinex passo 3}
 \label{fig:bitkinex4}
\end{figure}




il server in questo caso è cdav.azazelo.org (quando configurerete bitkinex sul vostro sistema dovrete inserire i parametri di configurazione fornitivi dalla scuola) notate come il protocollo scelto sia HTTP.

La configurazione è terminata, chiaramente vi verranno chieste password e username
prima della connessione.
Se tutto è andato a buon fine potete iniziare a navigare l'albero dei contenuti del sito.
Prima di effettuare il caricamento di file è necessario deselezionare un'opzione di verifica
trasferimento dati (a causa della rinomina automatica dei file contenenti spazi). Per questo
dal menu Data Source selezionate Properties.
Vi si aprirà la seguenti finestra di dialogo:

\begin{figure}[H]
 \centering
 \includegraphics[width=\textwidth]{./immagini/webdav/webdav5.png}
 % webdav5.png: 902x646 pixel, 96dpi, 23.86x17.09 cm, bb=
 \caption{Configurazione di BitKinex passo 4}
 \label{fig:bitkinex5}
\end{figure}



Dovete selezionare la voce Transfers come nell'immagine sopra. Togliete la spunta da
Inherit properties from the parent node (Http/WebDav) e cliccate sul menù a tendina Post
transfer data integrity checks:

\begin{figure}[H]
 \centering
 \includegraphics[width=\textwidth]{./immagini/webdav/webdav6.png}
 % webdav6.png: 585x384 pixel, 96dpi, 15.48x10.16 cm, bb=
 \caption{Configurazione di BitKinex passo 5}
 \label{fig:bitkinex}
\end{figure}





Togliete ora la spunta da Upload:destination file exists. Ora siete pronti ad utilizzare il
backend Webdav del sito

\begin{figure}[H]
 \centering
 \includegraphics[width=\textwidth]{./immagini/webdav/webdav7.png}
 % webdav7.png: 594x379 pixel, 96dpi, 15.71x10.03 cm, bb=
 \caption{Configurazione di BitKinex passo 6}
 \label{fig:bitkinex6}
\end{figure}


\subsection{Windows::WebDrive}


WebDrive è un client WebDav in grado di mappare una risorsa WebDav Remota su un device di rete accessibile tramite il pannello di controllo di Windows. Potete scaricare il programma all'indirizzo http://www.webdrive.com. L'installazione è molto semplice, è sufficiente fare doppio click sull'installatore appena scaricato e seguire le indicazioni a video.

\begin{figure}[H]
 \centering
 \includegraphics[width=0.75\textwidth]{./immagini/webdav/webdrive1.png}
 % webdrive1.png: 503x382 pixel, 72dpi, 17.74x13.48 cm, bb=
 \caption{Installazione di Webdrive}
 \label{fig:webdrive1}
\end{figure}

Dopo aver eseguito l'installazione del programma eseguitelo per iniziare la configurazione della connessione al server webdav scolastico:

\begin{figure}[H]
 \centering
 \includegraphics[width=0.75\textwidth]{./immagini/webdav/webdrive2.png}
 % webdrive2.png: 629x430 pixel, 72dpi, 22.19x15.17 cm, bb=
 \caption{Configurazione di WebDrive. Fare click su \textsl{New Site} per iniziare}
 \label{fig:webdrive2}
\end{figure}

Dopo aver cliccato su \textsl{New Site} appare una procedura guidata per la configurazione della connessione. Come prima cosa ci viene chiesto un nome da dare alla connessione ed una lettera di unità da assegnarle. Attenzione: scegliete una lettera libera ovvero non utilizzata da nessun altro device (chiavette, hdd esterni etc.).
\begin{figure}[H]
 \centering
 \includegraphics[width=0.75\textwidth]{./immagini/webdav/webdrive3.png}
 % webdrive3.png: 439x385 pixel, 72dpi, 15.49x13.58 cm, bb=
 \caption{Configurazione di Webdrive: scelta del nome e della lettera di unità}
 \label{fig:webdrive3}
\end{figure}
La procedura guidata procede con la scelta del protocollo che nel nostro caso sarà chiaramente webdav:

\begin{figure}[H]
 \centering
 \includegraphics[width=0.65\textwidth]{./immagini/webdav/webdrive4.png}
 % webdrive3.png: 439x385 pixel, 72dpi, 15.49x13.58 cm, bb=
 \caption{Configurazione di Webdrive:scelta del protocollo}
 \label{fig:webdrive4}
\end{figure}


ci viene poi chiesto di inserire l'indirizzo del server webdav. In questo campo inserirete l'url del server fornitavi dalla scuola:
\begin{figure}[H]
 \centering
 \includegraphics[width=0.65\textwidth]{./immagini/webdav/webdrive5.png}
 % webdrive3.png: 439x385 pixel, 72dpi, 15.49x13.58 cm, bb=
 \caption{Configurazione di Webdrive: scelta del server}
 \label{fig:webdrive5}
\end{figure}

Per non dover inserire ad ogni utilizzo username e password WebDrive è in grado di memorizzare i dati di autenticazione:

\begin{figure}[H]
 \centering
 \includegraphics[width=0.65\textwidth]{./immagini/webdav/webdrive6.png}
 % webdrive3.png: 439x385 pixel, 72dpi, 15.49x13.58 cm, bb=
 \caption{Configurazione di Webdrive: inserimento username e password}
 \label{fig:webdrive6}
\end{figure}

Se la nostra macchina è costantemente collegata ad internet possiamo scegliere la connessione automatica dell'unita all'avvio di Windows:
\begin{figure}[H]
 \centering
 \includegraphics[width=0.65\textwidth]{./immagini/webdav/webdrive7.png}
 % webdrive3.png: 439x385 pixel, 72dpi, 15.49x13.58 cm, bb=
 \caption{Configurazione di Webdrive: connessione automatica}
 \label{fig:webdrive7}
\end{figure}



\subsection{KDE::Konqueror 4}
Per utilizzare il servizio WebDav in Kde4 è sufficiente aprire konqueror e nella barra degli indirizzi digitare l'url del sito:
\begin{verbatim}
 webdav://webdav.miosito.net
\end{verbatim}

\begin{figure}[H]
 \centering
 \includegraphics[width=0.8\textwidth]{./immagini/webdav/konqueror.png}
 % konqueror.png: 1030x621 pixel, 72dpi, 36.34x21.91 cm, bb=
 \caption{Utilizzo di webdav tramite konqueror}
 \label{fig:konqueror1}
\end{figure}


\subsection{Unix Like::Cadaver}

Cadaver è un client  testuale disponibile sulla quasi totalità dei sistemi Unix Like e supporta pienamente tutte le funzionalità del sito.

\begin{verbatim}
user@amrael:~$ cadaver webdav.miosito.net
\end{verbatim}


\subsection{Linux::FuseDav}

Usando i moduli fuse disponibili per il kernel Linux è possibile montare un server webdav come un device a blocchi. 
\begin{verbatim}
user@amrael:~$ fusedav http://webdav.miosito.net/ /tmp/test
\end{verbatim}
\section{Utilizzo del sito via WebDav}
