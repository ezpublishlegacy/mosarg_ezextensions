\chapter[Registrazione]{La registrazione degli utenti}

Attualmente il sito prevede la presenza di quattro classi di utenti:

\begin{itemize}
 \item Genitori
\item Professori
\item Ata
\item Alunni
\end{itemize}


Dopo aver selezionato il collegamento \emph{Registrati} l'utente ha la possibilità di scegliere tra le classi prima illustrate quella più affine.

\begin{figure}[H]
 \centering
 \includegraphics[width=0.8\textwidth]{./immagini/registrazione/scelta_genere.png}
 % scelta_genere.png: 524x362 pixel, 72dpi, 18.49x12.77 cm, bb=
 \caption{Selezione del tipo di utente}
 \label{fig:reg_sel}
\end{figure}
Per ogni classe di utenti verrà presentata dal sistema una diversa form dedicata all'inserimento dei dati personali:
\begin{itemize}
 \item Genitori
\begin{itemize}
 \item Nome, Cognome, Dati account
\item Scuola frequentata dai figli
\item Figli frequentanti le varie scuole
\end{itemize}
\item Professori
\begin{itemize}
 \item Nome, Cognome, Dati account
\item Scuole di appartenenza
\item Classi di docenza
\item Sedi geografice di insegnamento
\end{itemize}
\item Ata
\begin{itemize}
 \item Nome, Cognome, Dati account
\item Scuole di appartenenza
\item Mansione
\end{itemize}
\item Alunni
\begin{itemize}
 \item Nome, Cognome, Dati account
\item Scuola frequentata
\end{itemize}
\end{itemize}


\section{Registrazione insegnante}

Durante la fase di registrazione l'insegnante è tenuto ad inserire:
\begin{description}
 \item[Nome e cognome] Il proprio nome e cognome con cui l'insegnante potrà essere individuato all'interno dell'istituto
\item[Account] Il sistema richiede il nome utente preferibilmente della forma \textbf{nome.cognome} e la password per accedere al sito. Si consiglia di utilizzare una password alfanumerica.
\item[Firma] Una frase da riportare nel proprio spazio utente
\item[Avatar] Una foto o un disegno rappresentanti l'utente
\item[Istituti] L'istituto o gli istituti in cui l'insegnante lavora
\item[Classi] Le classi di docenza del presente anno scolastico. Le scelte fatte durante il processo di registrazione circa le classi potranno essere modificate in qualsiasi momento
\item[Sede geografica] La sede o le sedi geografiche in cui l'insegnante lavora 
\end{description}

\begin{figure}[H]
 \centering
 \includegraphics[width=\textwidth]{./immagini/registrazione/registrazione_prof1.png}
 % registrazione_prof1.png: 1126x621 pixel, 72dpi, 39.72x21.91 cm, bb=
 \caption{Inserimento del nome cognome e dei dati account}
 \label{fig:reg_prof1}
\end{figure}
\begin{figure}[H]
 \centering
 \includegraphics[width=\textwidth]{./immagini/registrazione/registrazione_prof2.png}
 % registrazione_prof2.png: 855x687 pixel, 72dpi, 30.16x24.24 cm, bb=
 \caption{Scelta delle scuole}
 \label{fig:reg_prof2}
\end{figure}


\begin{figure}[H]
 \centering
 \includegraphics[width=\textwidth,bb=0 0 760 158]{./immagini/registrazione/registrazione_prof3.png}
 % registrazione_prof3.png: 760x158 pixel, 72dpi, 26.81x5.57 cm, bb=0 0 760 158
 \caption{Scelta delle materie e della sede geografica}
 \label{fig:reg_prof3}
\end{figure}
\section{Registrazione ata}

Per effettuare la registrazione un membro del personale Ata dovrà completare i seguenti campi:

\begin{description}
 \item[Nome e cognome] Il proprio nome e cognome con cui l'insegnante potrà essere individuato all'interno dell'istituto
\item[Account] Il sistema richiede il nome utente preferibilmente della forma \textbf{nome.cognome} e la password per accedere al sito. Si consiglia di utilizzare una password alfanumerica.
\item[Firma] Una frase da riportare nel proprio spazio utente
\item[Avatar] Una foto o un disegno rappresentanti l'utente
\item[Istituti] L'istituto o gli istituti in cui l'insegnante lavora
\item[Mansioni] Le mansioni svolte all'interno dell'istituto
\end{description}


 \begin{figure}[H]
 \centering
 \includegraphics[width=\textwidth]{./immagini/registrazione/ata_reg_1.png}
 % ata_reg_1.png: 763x156 pixel, 72dpi, 26.92x5.50 cm, bb=
 \caption{Scelta della mansione durante la registrazione}
 \label{fig:reg_ata1}
\end{figure}



\section{Registrazione genitori}

Per effettuare la registrazione i genitori dovranno complatare i seguenti campi:

\begin{description}
 \item[Nome e cognome] Il proprio nome e cognome con cui l'insegnante potrà essere individuato all'interno dell'istituto
\item[Account] Il sistema richiede il nome utente preferibilmente della forma \textbf{nome.cognome} e la password per accedere al sito. Si consiglia di utilizzare una password alfanumerica.
\item[Firma] Una frase da riportare nel proprio spazio utente
\item[Avatar] Una foto o un disegno rappresentanti l'utente
\item[Istituti] L'istituto o gli istituti in cui si trovano i figli
\item[Figli] Elenco dei figli con corrispondente scuola di appartenenza
\end{description}

\begin{figure}[H]
 \centering
 \includegraphics[width=\textwidth]{./immagini/registrazione/genitore_reg_1.png}
 % genitore_reg_1.png: 770x254 pixel, 72dpi, 27.16x8.96 cm, bb=
 \caption{Inserimento dei figli con relativa scuola di appartenenza}
 \label{fig:reg_gen1}
\end{figure}

\section{Registrazione Alunni}

Per effettuare la registrazione gli alunni dovranno complatare i seguenti campi:

\begin{description}
 \item[Nome e cognome] Il proprio nome e cognome con cui l'insegnante potrà essere individuato all'interno dell'istituto
\item[Account] Il sistema richiede il nome utente preferibilmente della forma \textbf{nome.cognome} e la password per accedere al sito. Si consiglia di utilizzare una password alfanumerica.
\item[Firma] Una frase da riportare nel proprio spazio utente
\item[Avatar] Una foto o un disegno rappresentanti l'utente
\item[Istituti] L'istituto in cui studi

\end{description}
\section{Procedure successiva alla registrazione}

Dopo aver concluso il processo ed inviato il formulario al server, vi verrà recapitata una e-mail (dovreste riceverla in pochi secondi). All'interno di questa e-mail troverete un link, seguitelo cliccandoci sopra con il mouse. Il vostro account viene quindi confermato. Come da indicazioni a monitor stampata l'email appena ricevuta e portatela a scuola, in questo modo avremo la certezza che soltanto i membri dell'istituto abbiano accesso alle parti riservate del sito. Non appena il vostro account sarà attivato riceverete automaticamente un'email.


